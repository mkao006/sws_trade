\documentclass[nojss]{jss}
\usepackage{url}
\usepackage[sc]{mathpazo}
\usepackage{geometry}
\geometry{verbose,tmargin=2.5cm,bmargin=2.5cm,lmargin=2.5cm,rmargin=2.5cm}
\setcounter{secnumdepth}{2}
\setcounter{tocdepth}{2}
\usepackage{breakurl}
\usepackage{hyperref}
\usepackage[ruled, vlined]{algorithm2e}
\usepackage{mathtools}
\usepackage{draftwatermark}
\usepackage{float}
\usepackage{placeins}
\usepackage{mathrsfs}
\usepackage{multirow}
%% \usepackage{mathbbm}
\DeclareMathOperator{\sgn}{sgn}
\DeclareMathOperator*{\argmax}{\arg\!\max}




\title{\bf Trade Processing with faoswsTrade}

\author{Michael. C. J. Kao\\ Food and Agriculture Organization \\ of
  the United Nations}

\Plainauthor{Michael. C. J. Kao} 

\Plaintitle{Trade Mirroring and Validation}

\Shorttitle{Trade Mirroring and Validation}

\Abstract{ 

  This manual provides the user with basic understanding of the
  functionality and the processing of trade data supported by the
  \pkg{faoswsTrade} package.


  The trade procedure consists of three steps, preprocessing and
  mirroring, conslidation and validation.

  
}

\Keywords{}
\Plainkeywords{}

\Address{
  Michael. C. J. Kao\\
  Economics and Social Statistics Division (ESS)\\
  Economic and Social Development Department (ES)\\
  Food and Agriculture Organization of the United Nations (FAO)\\
  Viale delle Terme di Caracalla 00153 Rome, Italy\\
  E-mail: \email{michael.kao@fao.org}\\
  URL: \url{https://github.com/mkao006/sws_flag}
}


\begin{document}

\section{Introduction}



\section{Modules}


\subsection{Pre-processing and Mirroring}

The initial step is to first clean any non-sensical data and perform
completion via mirroring. The processing of mirroring ensures that
un-reported trade are recorded whenever trading partner has reported
them.

The specific steps are listed below, we begin with the unprocessed
Comtrade raw data.

\begin{enumerate}
  \item Remove Self-trade.
  \item Remove inconsistent quantity, value pairs.
  \item Add re-trade to trade.
  \item Mirrored the data according to trading partner.
  \item Calculate unit value.
\end{enumerate}

\subsection{Consolidation}

After the data has been mirrored, we consolidated the data for the use
of the Food Balance Sheet.

In response to the request of team B and C, the values are not
balanced at this step. 

\begin{enumerate}
  \item Consolidated the data by aggregating according to reporting
    country and commodity by year.
  \item The country code is translated from Comtrade speciic M49
    system to the standard M49.
\end{enumerate}


\subsection{Validation}
The values from the completed trade flow are validated according to
the following rules. The rules are only applied to unit values, as no
rules are deem acceptable for quantity and values.

\begin{enumerate}
  \item Validation by mirrored value
  \item Validation by range - only for Primary commodities excluding livestock.
\end{enumerate}


\subsection{Balancing}

After the data has been validated, we create the balanced trade flow
for modelling purposes.

\begin{enumerate}
  \item Compute trade quantity wherever possible.
  \item Compute reliability index based solely on non-mirrored data.
  \item Update trade quantity based on reliability index.
  \item Update trade value based on quantity and unit value.
\end{enumerate}




%% \section{The algorithm}


%% The algortihm proceed as below
%% \begin{itemize}
%%   \item Mirror step:
%%   \begin{enumerate}
%%     \item Remove self trade
%%     \item Remove inconsistent trade quantity and value
%%     \item Add re-trade value and quantity to trade value and quantity
%%     \item Mirror trade value and quantity
%%     \item Calculate unit value
%%     \item Write mirrored data back to the data base as mirrored version
%%   \end{enumerate}
%%   \item Validation step:
%%   \begin{enumerate}
%%     \item Validate unit value by mirror value
%%     \item Validate unit value by range
%%     \item Calculate missing trade quantity if trade value and trade unit value exist
%%     \item Calculate reliability index
%%     \item Overwrite inconsistent trade quantity by reliability index
%%     \item Recalculate trade value based on trade quantity and trade unit value.
%%     \item Write validated data back to the data base as validated version.    
%%   \end{enumerate}
%%   \item Consolidation step:
%%   \begin{enumerate}
%%     \item Aggregation the trade quantity and value by reporting country.
%%   \end{enumerate}
%% \end{itemize}
  


%% Problems:

%% \begin{itemize}
%%   \item How to calculate the reliability index based only on
%%     non-mirrored value.
%%   \item How should the selection be if the reliability index is tied
%%     between two countries.
%%   \item Need to double check the validation of unit value by mirror
%%     function.
%% \end{itemize}



\end{document}
